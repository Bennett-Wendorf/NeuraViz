\documentclass{beamer}
\usetheme{Darmstadt}
\usepackage{latexsym,fancyhdr,amsmath,amsfonts,amsthm,dsfont}
\usepackage{tabularx}
\usepackage{amssymb}
\usepackage{graphicx}
\graphicspath{ {./res/} }
\usepackage{color,latexsym,fancyhdr,amsmath,amsfonts,dsfont,amssymb, amsthm}
%Information to be included in the title page:
\title{A Generalized Web-Based Application for Neural Network Visualization}
\author{Bennett Wendorf}
\institute{University of Wisconsin - La Crosse}
\date{April 25, 2023}
 
\begin{document}
 
\frame{\titlepage}
 
\begin{frame}
    \frametitle{1. Objective}
    \begin{itemize}
        \item Develop a web-based tool that will allow users to visualize neural networks
        \item Generalized to allow models from frameworks like TensorFlow, Keras, and PyTorch
    \end{itemize}
    \vspace{1cm}
    \centering
    \includegraphics[scale=.25, width=.2\textwidth]{TF_FullColor_Icon.png}\hfill
    \includegraphics[scale=.05, width=.2\textwidth]{768px-Keras_logo.png}\hfill
    \includegraphics[scale=.25, width=.2\textwidth]{pytorch-logo-dark.png}
\end{frame}
    
\begin{frame}
    \frametitle{2. Background}
    Existing Tools:
    \begin{itemize}
        \item TensorBoard
        \item Netron
        \item ENNUI
    \end{itemize}
    These tools lack support for multiple frameworks, different types of networks, and animation capabilities.
\end{frame}
 
\begin{frame}
    \frametitle{3. Primary Goals} 
    \begin{itemize}
        \item Visualize networks from TensorFlow, Keras, and PyTorch models
        \item Visualize simple feed-forward and convolutional neural networks
        \item Visualize the overall structure, plus details like weights and biases 
    \end{itemize}
    \centering
    \includegraphics{simple_NN.jpg}
\end{frame}

\begin{frame}
    \frametitle{4. Stretch Goals} 
    \begin{itemize}
        \item Animations of the network's training
        \item Inputs to generate custom networks and code for them
        \item Export visualizations and animations to tikz graphs for \LaTeX 
        \item Support for recurrent neural networks
        \item Support for ML.NET and scikit-learn models
    \end{itemize}
    \centering
    \includegraphics[scale=.20]{large_NN.png}
\end{frame}

\begin{frame}
    \frametitle{5. Challenges}
    \begin{itemize}
        \item Model parser complexity
        \item Graph generation and visualization complexity
        \item Complex network types (convolutional, etc.) are needed to be competitive
        \item Code export requires a deep understanding of the framework APIs
        \item Animations are challenging, especially when generating programmatically
    \end{itemize}
\end{frame}

\begin{frame}
    \frametitle{6. Schedule}
    \begin{tabular}{ |m{10em}|m{5em}|m{5em}|m{3em}| }
        \hline
        \textbf{Phase} & \textbf{From} & \textbf{To} & \textbf{Credits} \\  
        \hline
        Develop requirements document and problem analysis & Sept 1, 2023 & Sept 31, 2023 & 1 \\
        \hline
        Produce MVP & Oct 1, 2023 & Dec 31, 2023 & 5 \\
        \hline
        Add stretch features & Jan 1, 2024 & Feb 31, 2024 & 3 \\
        \hline
        Refine and test & Mar 1, 2024 & Mar 31, 2024 & 2 \\
        \hline
        Demonstration and project report & Apr 1, 2024 & May 10, 2024 & 1 \\
        \hline
    \end{tabular}
    \begin{tabular}{ m{10em} m{5em} m{5em} m{3em} }
        & & Total: & 12 \\
    \end{tabular}
\end{frame}

\begin{frame}
    \frametitle{7. Resources}
    \begin{itemize}
        \item Personal computer
        \item Hosting as needed
        \item Popular datasets (MNIST, Iris, etc.)
    \end{itemize}
\end{frame}

\begin{frame}
    \frametitle{Thank you}
    Questions
    \centering
\end{frame}

\begin{frame}
    \frametitle{Sources}
    \begin{itemize}
        \item \url{https://www.tensorflow.org/}
        \item \url{https://keras.io/}
        \item \url{https://pytorch.org/}
        \item \url{https://peerj.com/articles/cs-344/}
        \item \url{https://cdn-images-1.medium.com/v2/resize:fit:1200/1*-teDpAIho_nzNShRswkfrQ.gif}
    \end{itemize}
\end{frame}

\end{document}
