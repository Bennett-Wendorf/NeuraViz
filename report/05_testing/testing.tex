\section{Testing}
\label{sec:Testing}

\subsection{Overview} 
Per the scrum software design process, software verification and validation were performed continuously throughout the development process. To accommodate this and ensure that thorough testing was performed, a \textit{Testing Required} status column was added to the project board. Once projects finished development, they were moved to this column to ensure that testing was completed before the project was considered complete.

\subsection{Unit Testing}
During development, a large amount of unit testing was defined and performed to support the continuous and thorough testing process that was defined at the outset of the project. Due to the small size of the development team, test cases were both written and executed by the developer. Test processes that were used during the development of NeuraViz can be broken down into two major categories: frontend or client testing and backend or server testing.

\subsubsection{Frontend Testing}
The frontend of NeuraViz was tested using a combination of automated and manual unit testing. Testing a user interface is a challenge due to its primarily visual nature, but tools exist to help automate this process. For this application, the automated testing platform Playwright \cite{playwright} was used to develop test cases that could be executed on each individual project and at any time. Automated testing in this manner is especially useful to ensure that test cases are performed both consistently and repeatably, with minimal human error. 

Playwright test cases work by defining actions that should be taken in the user interface, and asserting various results. These results can include a variety of things, such as the presence of certain elements on the page, that a certain number of elements exist, or text on an element. Test cases are written directly in JavaScript and by default run in an automated headless browser so as not to interfere with other work that is being done. An example of a test case that was written for NeuraViz is shown in Listing \ref{lst:playwright-example}. This test case ensures that clicking the model upload button with no model file selected leads to the validation text staying red and saying "Model is not valid". On line 3 an example can be seen of an action being taken; in this case, the button with the name "Upload" is clicked. On lines 4-6, assertions are made about the state of the page after the action is taken.

\begin{center}
    \begin{lstlisting}[language=JavaScript, float=*htb, caption={Playwright Test Case Example}, label={lst:playwright-example}]
    test('Upload no model: Sidebar', async ({ page }) => {
        await page.goto('/');
        await page.getByRole('button', 
            { name: 'Upload' }).click();
        await expect(page.locator('#model-upload'))
            .toHaveValue('');
        await expect(page.locator('#model-validation'))
            .toHaveText('Model is not valid');
        await expect(page.locator('#model-validation'))
            .toHaveClass(/text-error/);
    });
    \end{lstlisting}
\end{center}

\subsubsection{Backend Testing}
% TODO: Add backend testing

\subsection{Regression Testing}
% TODO: Add Regression Testing