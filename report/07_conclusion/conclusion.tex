\section{Conclusion}																	
\label{sec:Conclusion}

\subsection{Overview} 
The NeuraViz project has successfully accomplished all of its primary goals in creating a user-friendly web application for visualizing artificial neural network architectures. It provides its users with the capability to upload pre-trained models from both Pytorch and Keras, visualize their structure, and export the visualizations to both SVG and LaTeX.

\subsection{Challenges}
A number of challenges and roadblocks were encountered throughout the development process, including a large amount of research into each framework, and personal time management. Each of the machine learning frameworks used have complex internal data structures that they use to model neural networks and be able to run inferences and training on them. However, these data structures are generally only designed to be used internally, and are not well documented for external use. In addition, because neural network training and inferencing is complex, the objects used by the frameworks often contain a large amount of information that is not relevant to the visualization, and must be filtered out. To develop NeuraViz's parsing algorithm, extensive research was required to understand the internal data structures that are used by Pytorch and Keras, and determine where necessary common information could be extracted to produce a cohesive experience for the user, regardless of how they chose to create their model in the first place.

As with many projects, time management was a constant struggle. Balancing development of NeuraViz with other obligations, whether school, work, or personal, was a significant hinderance to the rate of development that could be performed on this project, and ultimately limited the feature set that was able to be completed. However, it was known at the outset that many of the goals of this project were extensive and would likely not be fully completed within the time frame of the project.

\subsection{Future Work}
% TODO: Add future work
Throughout the planning and development of this project, a number of additional potential features were identified that could be added to NeuraViz to improve its functionality and usability. These include additions such as adding support for more model types, visualizing more complex networks like recurrent or convolutional networks, and animating the visualization to show the flow of data through the network.

While Pytorch and Keras cover a substantial portion of the industry for creating, training, and deploying neural networks, there are a multitude of other frameworks that are used in various industries. To fully realize the potential that NeuraViz provides, ideally every major framework would be supported, allowing users to visualize their models in a consistent manner regardless of their chosen framework. This would require extensive research both into what frameworks are used in industry, and the exact internal structures they use to model neural networks. On top of that, maintaining the tool with the ever-increasing number of products on the market would be a monumental task. Hopefully with NeuraViz being open source and freely available, more developers will be able to contribute and help expand this list. 

Visualizing more complex network types was one of the initial goals of this project. However, development tended more toward a use-case in an educational environment, rather than professional. While recurrent and convolutional networks are still beneficial to teach, they are more complex and not as often taught in an academic capacity, and therefore were left out of the project. However, adding support for these types of networks would drastically increase the utility of NeuraViz's major function of providing a consistent representation of many different types of neural networks. 

With the tendency more toward education during this project's development, the ability to animate data through a smaller network would be a great addition. An instructor could show a visualization to a class, and step through each step of the inferencing process live, instantly seeing how changes in the input data affect the output. Implementing a feature like this would take a significant amount of work, so it was left out of the project scope for the time-being.