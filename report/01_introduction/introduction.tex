\section{Introduction}
\label{sec:Introduction}

\subsection{Overview} 
This project aims to develop a software system to visualize the architecture of artificial neural networks. Neural networks (NNs) are a class of machine learning models that are inspired by the structure and function of the human brain. They are composed of a large number of interconnected processing elements, called neurons, which work together to solve complex problems. The architecture of a neural network refers to the arrangement of neurons and the connections between them. In addition to the general structure of a network, the weights and biases that control its function can provide useful insight into the inner workings of the model. The final integral parts of the neural network architecture are the activation functions that govern how data propagates through the network. Visualizing the architecture of a neural network can help students and researchers understand the structure of the model, identify potential issues, and communicate the model to others. 
% TODO: Consider linking to external resources for neural network information

This project will develop a web application that allows users to upload pre-trained neural network models and generate visual representations of their architecture. The application will support models trained using popular machine learning frameworks such as PyTorch and Keras. The resulting graph structure will be visualized in a pannable and zoomable svg format that shows the ordering of neurons, biases on those neurons, the weights of the connections between them and activation functions on each layer.
% TODO: Consider adding examples of what these architectures might look like

The application will be designed to be user-friendly and accessible to a wide range of users, including students and researchers. It will be implemented using modern web technologies and will be deployed as a web service that can be accessed from any device with an internet connection. The project will follow best practices in software engineering, including requirements analysis, design, implementation, testing, and deployment. The resulting application will be a valuable tool for anyone working with neural networks, but will be primarily targeted toward post-secondary educational students learning about machine learning and neural networks for the first time.

\subsection{Background}
Neural networks are a notoriously difficult concept to understand, especially for those who are newer to the field of machine learning. The architecture of a neural network is a key component of understanding how the model functions, but it can be difficult to visualize and comprehend. There are a handful to tools available for visualizing neural network architectures, but they are often limited in their scope. For example, Google's TensorBoard is a popular tool, but only natively supports TensorFlow and Keras models. In researching existing tools such as TensorBoard, Netron, and ENNUI, it was found that none of them supported the same range of formats as NeuraViz. This project aims to develop a user-friendly tool that is accessible to a wide range of users, including students and researchers who are new to the field of machine learning, and to support a wider range of model formats than other offerings.

\subsection{Goals}
NeuraViz is designed specifically with user-friendliness and simplicity in mind. The minimal interface focuses on the visualizes of the neural networks themselves, with minimal distractions. The use of color and shape in the visualizations will help to make the network architecture more understandable at a glance, quickly identifying neurons and connections that are the most important. Where needed, users can also zoom in or out an click on elements to see more detailed information.

Portability is another key goal of the development. Modern web technologies ensure the application is accessible from a range of devices, though it is most optimized for a desktop or laptop experience. Since the application is deployed as a web application, users don't need to install software on their own machines, or even sign in to be able to use the tool. To enhance this ability further, graph representations can also be exported as raw SVG or in tikz format for use within LaTeX documents. 
